\def\mytitle{BCD to GRAY Conversion}
\def\myauthor{Valmeekam Navya}
\def\contact{}
\def\mymodule{Future Wireless Communication (FWC)}
\documentclass[10pt, a4paper]{article}
\usepackage[a4paper,outer=1.5cm,inner=1.5cm,top=1.75cm,bottom=1.5cm]{geometry}
\twocolumn
\usepackage{graphicx}
\graphicspath{{./images/}}
\usepackage[colorlinks,linkcolor={black},citecolor={blue!80!black},urlcolor={blue!80!black}]{hyperref}
\usepackage[parfill]{parskip}
\usepackage{lmodern}
\usepackage{tikz}
%\documentclass[tikz, border=2mm]{standalone}
\usepackage{karnaugh-map}
%\documentclass{article}
\usepackage{tabularx}
\usepackage{circuitikz}
\usetikzlibrary{calc}
\usepackage{enumitem}

\renewcommand*\familydefault{\sfdefault}
\usepackage{watermark}
\usepackage{lipsum}
\usepackage{xcolor}
\usepackage{listings}
\usepackage{float}
\usepackage{titlesec}
\usepackage{kvmap}
       \usepackage[latin1]{inputenc}
       \usepackage{color}
       \usepackage{array}
       \usepackage{longtable}
       \usepackage{calc}
       \usepackage{multirow}
       \usepackage{hhline}
       \usepackage{ifthen}

\titlespacing{\subsection}{1pt}{\parskip}{3pt}
\titlespacing{\subsubsection}{0pt}{\parskip}{-\parskip}
\titlespacing{\paragraph}{0pt}{\parskip}{\parskip}
\newcommand{\figuremacro}[5]{
    
}

\lstset{
frame=single, 
breaklines=true,
columns=fullflexible
}

\def\ifundefined#1{\expandafter\ifx\csname#1\endcsname\relax}
\ifundefined{inputGnumericTable}
\def\gnumericTableEnd{\end{document}}
\else
   \def\gnumericTableEnd{}
\fi
\providecommand{\gnumericmathit}[1]{#1} 
\providecommand{\gnumericPB}[1]%
{\let\gnumericTemp=\\#1\let\\=\gnumericTemp\hspace{0pt}}
 \ifundefined{gnumericTableWidthDefined}
        \newlength{\gnumericTableWidth}
        \newlength{\gnumericTableWidthComplete}
        \newlength{\gnumericMultiRowLength}
        \global\def\gnumericTableWidthDefined{}
 \fi
 \ifthenelse{\isundefined{\languageshorthands}}{}{\languageshorthands{english}}
\providecommand\gnumbox{\makebox[0pt]}
\setlength{\bigstrutjot}{\jot}
\setlength{\extrarowheight}{\doublerulesep}
\setlongtables

\setlength\gnumericTableWidth{%
	98pt+%
	118pt+%
0pt}
\def\gumericNumCols{2}
\setlength\gnumericTableWidthComplete{\gnumericTableWidth+%
         \tabcolsep*\gumericNumCols*2+\arrayrulewidth*\gumericNumCols}
\ifthenelse{\lengthtest{\gnumericTableWidthComplete > \linewidth}}%
         {\def\gnumericScale{1*\ratio{\linewidth-%
                        \tabcolsep*\gumericNumCols*2-%
                        \arrayrulewidth*\gumericNumCols}%
{\gnumericTableWidth}}}%
{\def\gnumericScale{1}}

\ifthenelse{\isundefined{\gnumericColA}}{\newlength{\gnumericColA}}{}\settowidth{\gnumericColA}{\begin{tabular}{@{}p{98pt*\gnumericScale}@{}}x\end{tabular}}
\ifthenelse{\isundefined{\gnumericColB}}{\newlength{\gnumericColB}}{}\settowidth{\gnumericColB}{\begin{tabular}{@{}p{118pt*\gnumericScale}@{}}x\end{tabular}}

%\thiswatermark{\centering \put(181,-119.0){\includegraphics[scale=0.13]{iith_logo3}} }
\title{\mytitle}
\author{\myauthor\hspace{1em}\\\contact\\FWC22039}
\begin{document}
	\maketitle
	\tableofcontents
	\begin{abstract}
	    This manual explains BCD to GRAY code conversion by finding boolean equations. \\
	   \end{abstract}
	  \section{BCD to GRAY Conversion}
The BCD to GRAY code converter takes the numbers 0, 1, . . . , 9 in binary as inputs and generates the converted number as output. Make connections as shown in table 1.

\textbf{Problem : -}
Implement BCD to GRAY conversion

   \section{Implementation}
 \textbf{Connections :-}
\begin{tabular}{|c|c|c|c|c|c|c|c|c|c|}
\hline
\textbf{Arduino} & 2 & 3 & 4 & 5 & 6 & 7 & 8  \\
\hline
\textbf{Display} & {a} & {b} & {c} & {d} & {e} & {f} & {g} \\
\hline
\end{tabular}
	
   \section{Karnaugh Map}
Using Boolean logic or kmaps, G0, G1, G2, G3 in the truth table can be expressed in terms of the inputs A,B,C,D 
\newline
\centering
\begin{kvmap}
\begin{kvmatrix}{C,D,A,B}
0&1&0&1 \\
0&1&0&1 \\
0&0&0&0 \\
0&1&0&0 \\
\end{kvmatrix}
\bundle[color=red]{1}{0}{1}{1}
\bundle[color=red]{3}{0}{3}{1}
\bundle[color=red]{1}{3}{1}{3}
\end{kvmap}
\newline
Kmap for G0
\begin{equation}
G0=A'C'D+A'CD'+AB'C'D 
\end{equation}
\begin{center}
\begin{kvmap}
\begin{kvmatrix}{C,D,A,B}
0&0&1&1 \\
1&1&0&0 \\
0&0&0&0 \\
0&0&0&0 \\
\end{kvmatrix}
\bundle[color=red]{2}{0}{3}{0}
\bundle[color=red]{0}{1}{1}{1}
\end{kvmap}\\
kmap for G1
\begin{equation}
G1=A'BC'+A'B'C 
\end{equation}
\end{center}
\begin{center}
\begin{kvmap}
\begin{kvmatrix}{C,D,A,B}
0&0&0&0 \\
1&1&1&1 \\
0&0&0&0 \\
1&1&0&0 \\
\end{kvmatrix}
\bundle[color=red]{0}{1}{1}{1}
\bundle[color=red]{0}{3}{1}{3}
\end{kvmap}
\end{center}
kmap for G2
\begin{equation}
G2=A'B+AB'C'   
\end{equation} \
\begin{center}
\begin{kvmap}
\begin{kvmatrix}{c,D,A,B}
0&0&0&0 \\
0&0&0&0 \\
0&0&0&0 \\
1&1&0&0 \\
\end{kvmatrix}
\bundle[color=red]{0}{3}{1}{3}
\end{kvmap}
\end{center}
Kmap for G3
\begin{equation}
G3=AB'C' 
\end{equation}
\newpage
Using Boolean logic or kmaps, a,b,c,d,e,f,g in the truth table can be expressed in terms of G0,G1,G2,G3  as:
\newline
\begin{equation}
a=G0'G1'G2G3'+G0G1'G2'G3'
\end{equation}
\begin{equation}
b=G0'G1'G2G3+G0'G1G2G3'+
G0G1'G2G3'
\end{equation}
\begin{equation}
c=G0'G1G2'G3'+G0'G1'G2G3
\end{equation}
\begin{equation}
d=G0'G1'G2G3'+G0G1G2G3'
+G0G1'G2'G3'
\end{equation}
\newline

\begin{kvmap}
\begin{kvmatrix}{G2,G3,G0,G1}
0&0&0&1 \\
0&0&0&0 \\
1&0&0&1 \\
1&0&0&1 \\
\end{kvmatrix}
\bundle[color=red,invert=true,reducespace=2pt,overlapmargins=6pt]{0}{2}{3}{3}
\bundle[color=cyan,invert=true,reducespace=2pt,overlapmargins=6pt]{3}{0}{3}{3}
\end{kvmap}\\
Kmap for e
\begin{equation}
    e=G0G3'+G0G2G3'
\end{equation}
\begin{kvmap}
\begin{kvmatrix}{G2,G3,G0,G1}
0&0&0&0 \\
1&0&0&0 \\
1&0&0&1 \\
1&0&0&0 \\
\end{kvmatrix}
\bundle[color=cyan,invert=true,reducespace=2pt,overlapmargins=6pt]{0}{2}{3}{2}
\bundle[color=red]{0}{1}{0}{2}
\bundle[color=black]{0}{2}{0}{3}
\end{kvmap}
\newline
Kmap for f
\begin{equation}
f=G0G2'G3'+G1G2'G3'+G0G1G3'
\end{equation}
\begin{kvmap}
\begin{kvmatrix}{G2,G3,G0,G1}
1&0&0&0 \\
0&0&0&0 \\
0&0&0&1 \\
1&0&0&0 \\
\end{kvmatrix}
\bundle[color=cyan,invert=true,reducespace=2pt,overlapmargins=6pt]{0}{0}{0}{3}
\bundle[color=red]{3}{2}{3}{2}
\end{kvmap}
\newline
Kmap for g
\begin{equation}
g=G1'G2'G3'+G1'G2G3+G0G1G2G3'
\end{equation}

\subsection{The steps for implementation:}
\begin{enumerate}
\item Connect the USB-UART pins to the Vaman ESP32 pins according to Table 

\begin{tabular}[c]{%
	b{\gnumericColA}%
	b{\gnumericColB}%
	}
\hhline{|-|-}
	 \multicolumn{1}{|p{\gnumericColA}|}%
	{\gnumericPB{\centering}\gnumbox{{\color[rgb]{0.79,0.13,0.12} VAMAN LC PINS}}}
	&\multicolumn{1}{p{\gnumericColB}|}%
	{\gnumericPB{\centering}\gnumbox{{\color[rgb]{0.79,0.13,0.12} UART PINS}}}
\\
\hhline{|--|}
	 \multicolumn{1}{|p{\gnumericColA}|}%
	{\gnumericPB{\centering}\gnumbox{GND}}
	&\multicolumn{1}{p{\gnumericColB}|}%
	{\gnumericPB{\centering}\gnumbox{GND}}
\\
\hhline{|--|}
	 \multicolumn{1}{|p{\gnumericColA}|}%
	{\gnumericPB{\centering}\gnumbox{ENB}}
	&\multicolumn{1}{p{\gnumericColB}|}%
	{\gnumericPB{\centering}\gnumbox{ENB}}
\\
\hhline{|--|}
	 \multicolumn{1}{|p{\gnumericColA}|}%
	{\gnumericPB{\centering}\gnumbox{TXD0}}
	&\multicolumn{1}{p{\gnumericColB}|}%
	{\gnumericPB{\centering}\gnumbox{RXD}}
\\
\hhline{|--|}
	 \multicolumn{1}{|p{\gnumericColA}|}%
	{\gnumericPB{\centering}\gnumbox{RXD0}}
	&\multicolumn{1}{p{\gnumericColB}|}%
	{\gnumericPB{\centering}\gnumbox{TXD}}
\\
\hhline{|--|}
	 \multicolumn{1}{|p{\gnumericColA}|}%
	{\gnumericPB{\centering}\gnumbox{0}}
	&\multicolumn{1}{p{\gnumericColB}|}%
	{\gnumericPB{\centering}\gnumbox{IO0}}
\\
\hhline{|--|}
	 \multicolumn{1}{|p{\gnumericColA}|}%
	{\gnumericPB{\centering}\gnumbox{5V}}
	&\multicolumn{1}{p{\gnumericColB}|}%
	{\gnumericPB{\centering}\gnumbox{5V}}
\\
\hhline{|-|-|}
\end{tabular}
 \item Flash the following setup code through USB-UART using laptop
\begin{center}
\fbox{\parbox{8cm}{\url{https://github.com/NavyaValmeekam/FWC/tree/main/IOT/codes/setup/src/main.cpp}}}
\end{center}
\begin{center}
\end{center}
\begin{lstlisting}
svn co https://github.com/NavyaValmeekam/FWC/tree/main/IOT/codes/setup
cd  setup
pio run
pio run -t upload
\end{lstlisting}

after entering your wifi username and password (in quotes below)
\begin{lstlisting}
#define STASSID "..." // Add your network credentials
#define STAPSK  "..."
\end{lstlisting}
in src/main.cpp file
\item You can notice that vaman will be connnected to the network credentials provided above.Connect your laptop to the same network ,You should be able to find the ip address of your vaman-esp on laptop using 
\begin{lstlisting}
ifconfig
nmap -sn 192.168.6.1/24
\end{lstlisting}
where your computer's ip address is the output of ifconfig and given by 192.168.6.x
\item Login to termux-ubuntu on the android device and execute the following commands:
\begin{lstlisting}
proot-distro login debian
cd  /data/data/com.termux/files/home/
mkdir iot
svn co https://github.com/NavyaValmeekam/FWC/tree/main/IOT/codes/ota
cd codes
\end{lstlisting}
\item Assuming that the username is krishna and password is 123, flash the following code wirelessly
\begin{center}
\fbox{\parbox{8cm}{\url{https://github.com/NavyaValmeekam/FWC/blob/main/IOT/codes/ota/src/main.cpp}}}
\end{center}
through 
\begin{lstlisting}
pio run 
pio run -t nobuild -t upload --upload-port ip_addres_of_esp
\end{lstlisting}
where you may replace the above ip address with the ip address of your vaman-esp.\\
\section{Truth Table}
Verify the output using below truth table.

\begin{tabular}{|c|c|c|c|c|c|c|c|c|c|c|c|c|c|c|}
\hline
\textbf{A} & {B} & {C} & {D} & {G3} & {G2} & {G1} & {G0} & {a} & {b} & {c} & {d} & {e} & {f} & {g} \\
\hline
0 & 0 & 0 & 0 & 0 & 0 & 0 & 0 & 0 & 0 & 0 & 0 & 0 & 0 & 1 \\
\hline
0 & 0 & 0 & 1 & 0 & 0 & 0 & 1 & 1 & 0 & 0 & 1 & 1 & 1 & 1 \\
\hline
0 & 0 & 1 & 0 & 0 & 0 & 1 & 1 & 0 & 0 & 0 & 0 & 1 & 1 & 0 \\
\hline
0 & 0 & 1 & 1 & 0 & 0 & 1 & 0 & 0 & 0 & 1 & 0 & 0 & 1 & 0 \\
\hline
0 & 1 & 0 & 0 & 0 & 0 & 1 & 1 & 0 & 1 & 0 & 0 & 0 & 0 & 0 \\
\hline
0 & 1 & 0 & 1 & 0 & 1 & 1 & 1 & 0 & 0 & 0 & 1 & 1 & 1 & 1 \\
\hline
0 & 1 & 1 & 0 & 0 & 1 & 0 & 1 & 0 & 1 & 0 & 0 & 1 & 0 & 0 \\
\hline 
0 & 1 & 1 & 1 & 0 & 1 & 0 & 0 & 1 & 0 & 0 & 1 & 1 & 0 & 0 \\
\hline
1 & 0 & 0 & 0 & 1 & 1 & 0 & 0 & 0 & 1 & 1 & 0 & 0 & 0 & 1 \\
\hline
1 & 0 & 0 & 1 & 1 & 1 & 0 & 1 & 0 & 0 & 0 & 0 & 0 & 0 & 1 \\
\hline
\end{tabular}
\end{enumerate}
\end{document}
